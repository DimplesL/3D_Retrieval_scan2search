\section{Conclusion}

In summary, this project build a 3D retrieval system which provides a ``scan to search'' solution for manufacturing components. A 3D bilateral filter is applied to reduce the scanning noise. Two types of rotation invariant shape descriptors are combined to describe the shape feature of 3D models. A series of tests indicate that the spherical harmonics shape descriptors can describe the shape feature of the model in detail. However, spherical harmonics may be affected by the sampling error in rasterization of the model. Therefore, the second descriptors distance histogram is used to roughly describe the shape feature again, so that to compensate the potential error of spherical harmonics. 

\section{Future works}

This system can still be improved in many aspects. The following are some topics that worth further investigation.

\begin{enumerate}
\item A better user interface: 

Since this project only implement a prototype retrieval system. The user interface is built in a MFC dialog. The user interface is not that user-friendly. So a better user interface can be built for fast rendering of meshes, as well as better appearance.

\item A fast way to compute spherical harmonics: 

Current spherical harmonics decomposition is time-consuming. A solution is to reduce the cut-off frequency from 32 to 16. However, such approach is just a palliative. A fast way to compute spherical harmonics should be investigated and applied to the system. For example, M. Mousa~\etal~\cite{mousa2006direct} provides a fast and accurate technique for computing spherical
harmonics.

\item New descriptors:

As it is mentioned in the Evaluation section (Section~\ref{sec:results_rotationinvarianttest}), current two types of descriptors (spherical harmonics and distance histogram) cannot detect interior rotation of a model. Thus a new type of descriptors can be created, which has rotation invariant property and will not lost information of interior rotation. 
\item Crawling the web for sizeable database:

Current models in the database are manually collected from the Internet. To build a sizeable database, manual collecting is impractical. Therefore a crawler can be implemented to collect specified manufacturing 3D models.  

\item Accelerating the matching speed in a sizable 3D database: 

Currently this system compute similarity of the query model with all the models in the database. It would be useful to build a high dimensional kd-tree for fast retrieval in a sizable database. 

\item Clustering datasets in the database for fast retrieval and precise presentation of results:

The database can be clustered into small subsets of data with similar shape feature. For example, a subset contains models with cylinder-like shape is built. If the input model is detected as cylinder-like shape, the system can quickly compute its similarity with subsets and find the subset with cylinder-like shape feature. In this way, the retrieval speed is accelerated. Moreover, the presentation of the candidate models would be more accurate, because more weights can be added to the models in the matched subset. This can also help to avoid irrelevant matching results. 

\end{enumerate}